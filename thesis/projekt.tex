%==============================================================================
% tento soubor pouzijte jako zaklad
% this file should be used as a base for the thesis
% (c) 2008 Michal Bidlo
% E-mail: bidlom AT fit vutbr cz
% Šablonu upravil / template edited by: Ing. Jaroslav Dytrych, dytrych@fit.vutbr.cz
%==============================================================================
% kodovaní: UTF-8 (zmena prikazem iconv, recode nebo cstocs)
% encoding: UTF-8 (you can change it by command iconv, recode or cstocs)
%------------------------------------------------------------------------------
% zpracování / processing: make, make pdf, make clean
%==============================================================================
% Soubory, které je nutné upravit: / Files which have to be edited:
%   projekt-20-literatura-bibliography.bib - literatura / bibliography
%   projekt-01-kapitoly-chapters.tex - obsah práce / the thesis content
%   projekt-30-prilohy-appendices.tex - přílohy / appendices
%==============================================================================
%\documentclass[]{fitthesis} % bez zadání - pro začátek práce, aby nebyl problém s překladem
\documentclass[english]{fitthesis} % without assignment - for the work start to avoid compilation problem
%\documentclass[zadani]{fitthesis} % odevzdani do wisu - odkazy jsou barevné
%\documentclass[english,zadani]{fitthesis} % for submission to the IS FIT - links are color
%\documentclass[zadani,print]{fitthesis} % pro tisk - odkazy jsou černé
%\documentclass[english,zadani,print]{fitthesis} % for the print - links are black
% * Je-li prace psana v anglickem jazyce, je zapotrebi u tridy pouzit 
%   parametr english nasledovne:
%   If thesis is written in english, it is necessary to use 
%   parameter english as follows:
%      \documentclass[english]{fitthesis}
% * Je-li prace psana ve slovenskem jazyce, je zapotrebi u tridy pouzit 
%   parametr slovak nasledovne:
%      \documentclass[slovak]{fitthesis}

% Základní balíčky jsou dole v souboru šablony fitthesis.cls
% Basic packages are at the bottom of template file fitthesis.cls
%zde muzeme vlozit vlastni balicky / you can place own packages here

%---rm---------------
\renewcommand{\rmdefault}{lmr}%zavede Latin Modern Roman jako rm / set Latin Modern Roman as rm
%---sf---------------
\renewcommand{\sfdefault}{qhv}%zavede TeX Gyre Heros jako sf
%---tt------------
\renewcommand{\ttdefault}{lmtt}% zavede Latin Modern tt jako tt

% vypne funkci šablony, která automaticky nahrazuje uvozovky,
% aby nebyly prováděny nevhodné náhrady v popisech API apod.
% disables function of the template which replaces quotation marks
% to avoid unnecessary replacements in the API descriptions etc.
\csdoublequotesoff

% =======================================================================
% balíček "hyperref" vytváří klikací odkazy v pdf, pokud tedy použijeme pdflatex
% problém je, že balíček hyperref musí být uveden jako poslední, takže nemůže
% být v šabloně
% "hyperref" package create clickable links in pdf if you are using pdflatex.
% Problem is that this package have to be introduced as the last one so it 
% can not be placed in the template file.
\ifWis
\ifx\pdfoutput\undefined % nejedeme pod pdflatexem / we are not using pdflatex
\else
  \usepackage{color}
  \usepackage[unicode,colorlinks,hyperindex,plainpages=false,pdftex]{hyperref}
  \definecolor{links}{rgb}{0.4,0.5,0}
  \definecolor{anchors}{rgb}{1,0,0}
  \def\AnchorColor{anchors}
  \def\LinkColor{links}
  \def\pdfBorderAttrs{/Border [0 0 0] }  % bez okrajů kolem odkazů / without margins around links
  \pdfcompresslevel=9
\fi
\else % pro tisk budou odkazy, na které se dá klikat, černé / for the print clickable links will be black
\ifx\pdfoutput\undefined % nejedeme pod pdflatexem / we are not using pdflatex
\else
  \usepackage{color}
  \usepackage[unicode,colorlinks,hyperindex,plainpages=false,pdftex,urlcolor=black,linkcolor=black,citecolor=black]{hyperref}
  \definecolor{links}{rgb}{0,0,0}
  \definecolor{anchors}{rgb}{0,0,0}
  \def\AnchorColor{anchors}
  \def\LinkColor{links}
  \def\pdfBorderAttrs{/Border [0 0 0] } % bez okrajů kolem odkazů / without margins around links
  \pdfcompresslevel=9
\fi
\fi
% Řešení problému, kdy klikací odkazy na obrázky vedou za obrázek
% This solves the problems with links which leads after the picture
\usepackage[all]{hypcap}

% Algorithms
\usepackage{algorithm}
\usepackage{algpseudocode}

% Informace o práci/projektu / Information about the thesis
%---------------------------------------------------------------------------
\projectinfo{
  %Prace / Thesis
  project=BP,            %typ prace BP/SP/DP/DR  / thesis type (SP = term project)
  year=2017,             %rok odevzdání / year of submission
  date=\today,           %datum odevzdani / submission date
  %Nazev prace / thesis title
  title.cs={Evoluční optimalizace analogových zesilovačů},  %nazev prace v cestine ci slovenstine (dle zadani) / thesis title in czech language (according to assignment)
  title.en={Evolutionary Analog Amplifier Optimisation}, %nazev prace v anglictine / thesis title in english
  %Autor / Author
  author={Marek Bielik},   %cele jmeno a prijmeni autora / full name and surname of the author
  author.name={Marek},   %jmeno autora (pro citaci) / author name (for reference) 
  author.surname={Bielik},   %prijmeni autora (pro citaci) / author surname (for reference) 
  %author.title.p=Bc., %titul pred jmenem (nepovinne) / title before the name (optional)
  %author.title.a=PhD, %titul za jmenem (nepovinne) / title after the name (optional)
  %Ustav / Department
  department=UPSY, % doplnte prislusnou zkratku dle ustavu na zadani: UPSY/UIFS/UITS/UPGM
  %                  fill in appropriate abbreviation of the department according to assignment: UPSY/UIFS/UITS/UPGM
  %Skolitel / supervisor
  supervisor=Michal Bidlo, %cele jmeno a prijmeni skolitele / full name and surname of the supervisor
  supervisor.name={Michal},   %jmeno skolitele (pro citaci) / supervisor name (for reference) 
  supervisor.surname={Bidlo},   %prijmeni skolitele (pro citaci) / supervisor surname (for reference) 
  supervisor.title.p=Ing.,   %titul pred jmenem (nepovinne) / title before the name (optional)
  supervisor.title.a={Ph.D.},    %titul za jmenem (nepovinne) / title after the name (optional)
  %Klicova slova, abstrakty, prohlaseni a podekovani je mozne definovat 
  %bud pomoci nasledujicich parametru nebo pomoci vyhrazenych maker (viz dale)
  %Keywords, abstracts, declaration and acknowledgement can be defined by following 
  %parameters or using dedicated macros (see below)
  %===========================================================================
  %Klicova slova / keywords
  %keywords.cs={Klíčová slova v českém jazyce.}, %klicova slova v ceskem ci slovenskem jazyce
  %                                              keywords in czech or slovak language
  %keywords.en={Klíčová slova v anglickém jazyce.}, %klicova slova v anglickem jazyce / keywords in english
  %Abstract
  %abstract.cs={Výtah (abstrakt) práce v českém jazyce.}, % abstrakt v ceskem ci slovenskem jazyce
  %                                                         abstract in czech or slovak language
  %abstract.en={Výtah (abstrakt) práce v anglickém jazyce.}, % abstrakt v anglickem jazyce / abstract in english
  %Prohlaseni / Declaration
  %declaration={Prohlašuji, že jsem tuto bakalářskou práci vypracoval samostatně pod vedením pana ...},
  %Podekovani (nepovinne) / Acknowledgement (optional)
  %acknowledgment={Zde je možné uvést poděkování vedoucímu práce a těm, kteří poskytli odbornou pomoc.} % nepovinne
  %acknowledgment={Here it is possible to express thanks to the supervisor and to the people which provided professional help.} % optional
}

%Abstrakt (cesky, slovensky ci anglicky) / Abstract (in czech, slovak or english)
\abstract[cs]{Do tohoto odstavce bude zapsán výtah (abstrakt) práce v českém (slovenském) jazyce.}
\abstract[en]{The standard way to design an amplifier is to calculate the values of its components by mathematical equations. The goal of this thesis is to use artificial intelligence and make the computer determine the values without the equations.}

%Klicova slova (cesky, slovensky ci anglicky) / Keywords (in czech, slovak or english)
\keywords[cs]{Sem budou zapsána jednotlivá klíčová slova v českém (slovenském) jazyce, oddělená čárkami.}
\keywords[en]{artificial intelligence, evolutionary computation, evolution strategy, SPICE, ngSPICE, analogue amplifier,}

%Prohlaseni (u anglicky psane prace anglicky, u slovensky psane prace slovensky)
%Declaration (for thesis in english should be in english)
%\declaration{Prohlašuji, že jsem tuto bakalářskou práci vypracoval samostatně pod vedením pana X...
%Další informace mi poskytli...
%Uvedl jsem všechny literární prameny a publikace, ze kterých jsem čerpal.}

\declaration{Hereby I declare that this bachelor's thesis was prepared as an original author’s work under the supervision of Ing. Michal Bidlo, Ph.D.
All the relevant information sources, which were used during preparation of this thesis, are properly cited and included in the list of references.}

%Podekovani (nepovinne, nejlepe v jazyce prace) / Acknowledgement (optional, ideally in the language of the thesis)
%\acknowledgment{V této sekci je možno uvést poděkování vedoucímu práce a těm, kteří poskytli odbornou pomoc
%(externí zadavatel, konzultant, apod.).}
\acknowledgment{Thank you all, my family, supervisor and Georgina.}

% řeší první/poslední řádek odstavce na předchozí/následující stránce
% solves first/last row of the paragraph on the previous/next page
\clubpenalty=10000
\widowpenalty=10000

\begin{document}
  % Vysazeni titulnich stran / Typesetting of the title pages
  % ----------------------------------------------
  \maketitle
  % Obsah
  % ----------------------------------------------
  \tableofcontents
  
  % Seznam obrazku a tabulek (pokud prace obsahuje velke mnozstvi obrazku, tak se to hodi)
  % List of figures and list of tables (if the thesis contains a lot of pictures, it is good)
\ifczech
  \renewcommand\listfigurename{Seznam obrázků}
\fi
\ifslovak
  \renewcommand\listfigurename{Zoznam obrázkov}
\fi

  % \listoffigures
\ifczech
  \renewcommand\listtablename{Seznam tabulek}
\fi
\ifslovak
  \renewcommand\listtablename{Zoznam tabuliek}
\fi

  % \listoftables 

  % vynechani stranky v oboustrannem rezimu
  % Skip the page in the two-sided mode
  \iftwoside
    \cleardoublepage
  \fi

  % Text prace / Thesis text
  % ----------------------------------------------
  %=========================================================================
% (c) Michal Bidlo, Bohuslav Křena, 2008

\include{Preface}

\chapter{Evolutionary algorithms}
\begin{quotation}
\textit{‘Owing to this struggle for life, variations, however slight and from whatever cause proceeding, if they be in any degree profitable to the individuals of a species, in their infinitely complex relations to other organic beings and to their physical conditions of life, will tend to the preservation of such individuals, and will generally be inherited by the offspring. The offspring, also, will thus have a better chance of surviving, for, of the many individuals of any species which are periodically born, but a small number can survive. I have called this principle, by which each slight variation, if useful, is preserved, by the term Natural Selection’.} \cite[p.~115]{Knuth}
\end{quotation}

In biological evolution, species are selected depending on their relative success in surviving and reproducing in the environment and mutations play a crucial role in this process as they are the key to the adaptation of the species. This concept has been used as a metaphor in computer science and it inspired the formation of a whole new area of artificial intelligence called Evolutionary Computation. This area mainly deals with an \textit{optimization problem} which we discuss in the following section.

\section{Optimization problem}
Consider the function $f: A \to \mathbb{R}$ from a set $A$ to the real numbers and we are looking for an element $x_0$ in $A$ such that $f(x_0) \leq f(x)$ or $f(x_0) \geq f(x)$ for all $x$ in $A$ (we are looking for either the global minimum or maximum of the function $f$). The domain $A$ of $f$ is called the \textit{search space} and the elements of $A$ are called \textit{candidate solutions}. The function $f$ can be called variously (an \textit{objective function}, a \textit{loss function}), in this text we will call the function $f$ a \textit{fitness function}. A candidate solution that gives us the global maximum or minimum of the fitness function is called an \textit{optimal solution}. The search space $A$ is usually in the form of $\mathbb{R}^n$ n-dimensional Euclidean space and the function $f$ is also n-dimensional.\\
A simple approach to find the optimal solution would be to scour the whole search space and try every candidate solution. This approach will always give us the optimal solution, but the search space can be so vast, that we are not able to try every candidate solution in reasonable time.\\
A more sophisticated approach would be to use the technique presented in algorithm \ref{evolutionary-algorithm}. The \textit{population} $P \subseteq A$ would be a subset of the search space $A$ and every \textit{individual} of the population would represent one element of $A$. The individuals can be also called \textit{chromosomes} in certain types of evolutionary algorithms. Every loop in the algorithm represents one \textit{generation} of individuals. This approach helps us to avoid those elements in the search space which wouldn't provide any good solution but the finding of the optimal solution is not guaranteed.\\
From the biological point of view, this algorithm can be seen as the driving force behind evolution and from the mathematical point of view it can be seen as a stochastic, derivative-free numerical method for finding global extrema of functions that are too hard or impossible to find analytically.
Biology uses this approach to create well adapted beings and humans can use this method to find optimal solutions to problems that are difficult to solve analytically.

\begin{algorithm}
	\caption{Evolutionary algorithm}\label{evolutionary-algorithm}
	\begin{algorithmic}[1]
		\State Create a \textit{population} P of randomly selected candidate solutions;
		\While{terminating condition}
		    \State \textit{(Selection)} Select the appropriate individuals (parents) for breeding from P;
		    \State \textit{(Mutation)} Generate new individuals (offspring) from the parents;
		    \State \textit{(Reproduction)} Replace some or all individuals in P with the new individuals;
		\EndWhile
	\end{algorithmic}
\end{algorithm}

\section{Evolution Strategies} \label{evolution-strategies}
\textit{Evolution strategies} (ES) is a family of stochastic optimization algorithms which belongs to the general class of evolutionary algorithms. It was created by Rechenberg and Schwefel in the early 60s and attracted a lot of attention due to its strong capability to solve real-valued engineering problems. ES typically uses a real-valued representation of chromosomes and it relies primarily on selection and mutation to drive the evolutionary process. ES also often implements \textit{self-adaptaion} of the parameters used for mutating the parents in the population during the evolution, which means that these parameters coevolve with the individuals. This feature is natural in evolution strategies and other evolutionary algorithms have adopted it as well over the last years.\\
Algorithm \ref{evolution-strategies} is a basic non-self-adaptive form of ES. Chromosomes in this algorithm are in the form of vectors of real numbers $x = (x_1,...,x_n) \in \mathbb{R}^n$. Every vector represents one solution of the search space in our problem domain and the goal of the algorithm is to find such vector that produces global extremum of our fitness function.\\
The terminating condition of the algorithm can have various forms which can be combined together:

 \begin{itemize}
    \item Wait until the algorithm finds a chromosome with a fitness function value that meets our requirements.
    \item Limit the total number of generations.
    \item Track the best chromosome in every generation and stop the evolution if the fitness doesn't change anymore.
 \end{itemize}

 The last two approaches are suitable when the algorithm gets stuck in a local optimum and doesn't leave it after a significant amount of time.\\
This form of the algorithm is also implemented in this thesis with various extensions in order to produce better results. These extensions are described and discussed in the rest of this chapter.

\begin{algorithm}[H]
\caption{Non-self-adaptive Evolution Strategies ($\mu + \lambda$)}\label{evolution-strategies}
\begin{algorithmic}[1]
    \State Randomly create an initial population $\{x^1,...,x^\mu\}$ of parent vectors, where each $\vec{x}^i$ is of the form $x^i = (x_1^i,...,x_n^i), i = 1,...,\mu$;
    \State Evaluate the fitness function of each chromosome;
    \While{terminating condition}
        \Repeat
            \State Randomly select a parent from $\{x^i : i = 1,...\mu\}$;
            \State Create a child vector by applying a mutation operator to the parent;
        \Until{$\lambda$ children are generated}
        \State Rank the $\mu + \lambda$ chromosomes (children and parents) from the best to worst;
        \State Select the best $\mu$ of these chromosomes to continue into the next generation;
    \EndWhile
\end{algorithmic}
\end{algorithm}

\section{Selection}
There are two main parameters in evolutionary strategies that are used to describe the type of the selection process:

 \begin{enumerate}
    \item \textbf{Parameter \boldmath$\mu$} specifies the number of chromosomes (parents) that are selected for reproduction
    \item \textbf{Parameter \boldmath$\lambda$} specifies the number of children that are produced in every generation.
 \end{enumerate}

There are also two types of selection:
\begin{center}
\begin{minipage}{.4\textwidth}
    \begin{enumerate}
        \item \boldmath$(\mu + \lambda)$\textbf{-ES} selection scheme
        \item $(\mu,\lambda)$\textbf{-ES} selection scheme
    \end{enumerate}
\end{minipage}
\end{center}

In the first scheme, denoted by $(\mu + \lambda)$-ES, both parents from the previous generation and newly produced children are mixed together and compete for survival in every generation where only the best $\mu$ chromosomes (parents for the next generation) are selected and get the chance to be reproduced. This scheme implements so called \textit{elitizm}, because if there is a chromosome with a very good fitness function, it can survive for many generations and it can be replaced only if a better individual occurs. This approach has a higher natural tendency to get stuck in a local optima than the second scheme.\\
In the second type of selection, denoted $(\mu,\lambda)$-ES, the parents are selected only from the set of children $\lambda$, so every chromosome lives only in one generation. Even a chromosome with a very good fitness function is discarted and replaced by a child with potencially worse fitness, so the convergence is not as strict as in the previous scheme, but it is easier to move away from a local optimum for this method.\\
Both schemes are implemented in this thesis with various values of $\mu$ and $\lambda$ parameters $(\mu \ll \lambda)$. Schemes such as $(1 + 1)$-ES and $(1, \lambda)$-ES are also possible ($(1 + 1)$-ES was mainly studied when evolution strategies was invented), but they show a slower convergency properties.

\section{Mutation} \label{mutation-section}
Chromosomes in evolution strategies are represented as vectors of real values, formula \ref{simple-chromosome-vector}.

\begin{equation} \label{simple-chromosome-vector}
\vec{x} = (x_1,...,x_n)
 \end{equation}

 These values correspond to the solution variables. Mutations can be performed in various ways, the simplest form is shown in formula \ref{simple-mutation} where $m$ is a real number drawn from a normal (Gaussian) distribution with the mean in $0$ and the stadard deviation $\sigma$ is chosen by the user. The parameter $\sigma$ represents the mutation step size and its value is crucial since it determines the effect of the mutation operator. The value should be kept relatively small to the problem domain in order not to perform large mutations too often. The parent chromosome is denoted by $\vec{x}(t)$ and $\vec{x}(t+1)$ denotes a child. It is also important to keep the values of the candidate solution $(x_1,...,x_n)$ in their domain interval and return it back every time the mutation shifts it out of the interval.

\begin{equation}
    m = N(0, \sigma)
\end{equation}
\begin{equation} \label{simple-mutation}
    \vec{x}(t+1) = \vec{x}(t) + m
\end{equation}

This approach ignores two important facts:
\begin{enumerate}
    \item When the search is close to the global optimum, it is appropriate to perform only subtle changes to the chromosomes so that the the algorithm will not leave the good region.
    \item Every dimension in the solution space can require different scaling, so that one dimension can require large mutation steps while another only small ones.
\end{enumerate}

Both these facts became crucial to the ability of the algorithm to find an optimal solution during the implementation, so they are discussed and examined in the following sections.

\subsection{Mutation with One Step Size}
In this version of mutation, we change the the vector $x$, formula \ref{one-step-chromosome-vector}.

\begin{equation} \label{one-step-chromosome-vector}
\vec{x} = ((x_1,...,x_n),\sigma)
\end{equation}

 $(x_1,...,x_n)$ is the original vector and $\sigma$ is a real-valued strategy parameter which controls the mutation process for every chromosome separately. The mutation process is described in formulas \ref{single-step-size-sigma} and \ref{single-step-size-mutation}. The constant $\tau$ is set by the user and it is inversely proportional to the square root of the problem size. It represents the learning rate of the algorithm. The $\sigma$ parameter is then mutated by multiplying with a variable with log-normal distribution, which ensures that $\sigma$ is always positive. The reasons for this form of mutation of $\sigma$ by a log-normal distribution are stated in \cite{x}. The newly mutated $\sigma$ is then used to mutate the solution values for the child where $i = 1,...,n$ are the $n$ elements making up one chromosome. It is important to keep the right order of the mutation - to mutate the value of $\sigma$ first and then mutate the vector's elements itself.

\begin{equation}
    \tau \propto \frac{1}{\sqrt{n}}
\end{equation}
\begin{equation} \label{single-step-size-sigma} \scalebox{1.3}{
    $\sigma(t+1) = \sigma(t) \cdot e^{\tau \cdot N(0,1)}$
}
\end{equation}
\begin{equation} \label{single-step-size-mutation}
    x_i(t+1) = x_i(t) + \sigma(t+1) \cdot N_i(0,1), \quad i = 1,...,n
\end{equation}

This approach allows the mutation step size to coevolve with the chromosomes, since the fitness function indirectly evaluates the suitability of the mutation. It is important to vary the step size during the evolution run as we want the algorithm to prefer longer steps when it is far away from the optima and small steps when it is close to the optima.

\subsection{Mutation with n-Step Size}
The fitness function surface can have different inclination in every dimension, so while one dimension requires large mutation steps another one might require only subtle ones. We can solve this problem by adding a strategy parameter $\sigma$ to every element of the chromosome's solution vector, formula \ref{n-step-size-chromosome-vector}.

 \begin{equation} \label{n-step-size-chromosome-vector}
    \vec{x} = ((x_1,...,x_n), \sigma_1,...,\sigma_n)
 \end{equation}

 The mutation process is then described in formulas \ref{n-step-size-sigma} and \ref{n-step-size-mutation}.

\begin{equation}
\tau' \propto \frac{1}{\sqrt{2\sqrt{n}}}
\end{equation}
\begin{equation} \label{n-step-size-sigma} \scalebox{1.3}{
    $\sigma_i(t+1) = \sigma_i(t) \cdot e^{\tau' \cdot N(0,1)' + \tau \cdot N_i(0,1)}$
}
\end{equation}
\begin{equation} \label{n-step-size-mutation}
    x_i(t+1) = x_i(t) + \sigma_i(t+1) \cdot N(0,1)_i, \quad i = 1,...,n
\end{equation}

Equation \ref{single-step-size-sigma} differs from \ref{n-step-size-sigma}. A simple modification of \ref{single-step-size-sigma} which we could do is shown in formula \ref{single-step-size-sigma-modification}. The reason why we add one more normally distributed variable in the actual algorithm (equation \ref{n-step-size-sigma}) is because we want keep the overall change of the mutability but we also want a finer granularity on the coordinate level.

\begin{equation}\label{single-step-size-sigma-modification}
\scalebox{1.3}{
    $\sigma_i(t+1) = \sigma_i(t) \cdot e^{\tau \cdot N_i(0,1)}$
}
\end{equation}

\include
\chapter{Evolutionary algorithms}
\begin{quotation}
	\textit{‘Owing to this struggle for life, variations, however slight and from whatever cause proceeding, if they be in any degree profitable to the individuals of a species, in their infinitely complex relations to other organic beings and to their physical conditions of life, will tend to the preservation of such individuals, and will generally be inherited by the offspring. The offspring, also, will thus have a better chance of surviving, for, of the many individuals of any species which are periodically born, but a small number can survive. I have called this principle, by which each slight variation, if useful, is preserved, by the term Natural Selection’.} (Darwin, 1859 [127], p. 115)
\end{quotation}

In biological evolution, species are selected depending on their relative success in surviving and reproducing in the environment and mutations play a crucial role in this process as they are the key to the adaptation of the species. This concept has been used as a metaphor in computer science and it inspired the formation of a whole new area of artificial intelligence called Evolutionary Computation. This area often deals with an \textit{optimization problem} which can be expressed in the following way: We have a function $f: A \to \mathbb{R}$ from a set $A$ to the real numbers and we are looking for an element $x_0$ in $A$ such that $f(x_0) \leq f(x)$ or $f(x_0) \geq f(x)$ for all $x$ in $A$ (we are looking for either the global minimum or maximum of the function $f$). The domain $A$ of $f$ is called the \textit{search space} and the elements of $A$ are called \textit{candidate solutions}. The function $f$ can be called variously (an \textit{objective function}, a \textit{loss function}), in this text we will call the function $f$ a \textit{fitness function}. A candidate solution that gives us the global maximum or minimum of the fitness function is called an \textit{optimal solution}.
A simple approach to find the optimal solution would be to scour the whole search space and try every candidate solution. This approach will always give us the optimal solution, but the search space can be so vast, that we are not able to try every candidate solution in reasonable time. 
A more sophisticated approach would be to use the technique presented in algorithm \ref{evolutionary-algorithm}. The \textit{population} would be a subset of $A$ (the search space) and every \textit{individual} of the population would represent one element of $A$. The individuals can be also called \textit{chromosomes} in certain types of evolutionary algorithms. This approach helps us to avoid thouse those elements in the search space which wouldn't provide any good solution, but the finding of the optimal solution is not guaranteed.
From the biological point of view, this algorithm can be seen as the driving force behind evolution and from the mathematical point of view it can be seen as a stochastic numerical method for finding global optima of functions that are too hard or impossible to find analytically.
\begin{algorithm}
	\caption{Evolutionary algorithm}\label{evolutionary-algorithm}
	\begin{algorithmic}[1]
		\State Create a population of randomly selected candidate solutions;
		\While{<optimal solution $\notin$ population>}
		\State (Selection) Select the appropriate individuals (parents) for breeding from the current population;
		\State (Mutation) Generate new individuals (offspring) from these parents;
		\State (Reproduction) Replace some or all of the current population with the new individuals;
		\EndWhile
	\end{algorithmic}
\end{algorithm}

\section{Evolutionary Strategies}


\section{Fitness function}
A fitness function provides us means by which we can determine how good the evolved solution is, so that we can distinguish the more appropriate candidates from the less appropriate ones. It is important to mention that the quality of this function is essential for every evolutionary algorithm, because the ability of the algorithm to find the most suitable solution is highly dependent on the ability of the fitness function to distinguish between the individuals. The value of the function is a positive real number in our case and the exact value is not very important, because the algorithm uses this value only for comparing the individuals between each other and we consider those individuals with the lower value as the better ones. There are three fitness functions implemented in this thesis.
\subsection{Best match with the reference solution}
This fitness function was used as the first solution in order to find out if the evolution has the ability to find the desired solution. It is based on the method of least squares which is used in regression analysis. 
In terms of fitness function, every individual is represented by a vector 

%=========================================================================
 % viz. obsah.tex / see obsah.tex

  % Pouzita literatura / Bibliography
  % ----------------------------------------------
\ifslovak
  \makeatletter
  \def\@openbib@code{\addcontentsline{toc}{chapter}{Literatúra}}
  \makeatother
  \bibliographystyle{bib-styles/czechiso}
\else
  \ifczech
    \makeatletter
    \def\@openbib@code{\addcontentsline{toc}{chapter}{Literatura}}
    \makeatother
    \bibliographystyle{bib-styles/czechiso}
  \else 
    \makeatletter
    \def\@openbib@code{\addcontentsline{toc}{chapter}{Bibliography}}
    \makeatother
    \bibliographystyle{bib-styles/englishiso}
  %  \bibliographystyle{alpha}
  \fi
\fi
  \begin{flushleft}
  \bibliography{projekt-20-literatura-bibliography}
  \end{flushleft}

  % vynechani stranky v oboustrannem rezimu
  % Skip the page in the two-sided mode
  \iftwoside
    \cleardoublepage
  \fi

  % Prilohy / Appendices
  % ---------------------------------------------
  \appendix
\ifczech
  \renewcommand{\appendixpagename}{Přílohy}
  \renewcommand{\appendixtocname}{Přílohy}
  \renewcommand{\appendixname}{Příloha}
\fi
\ifslovak
  \renewcommand{\appendixpagename}{Prílohy}
  \renewcommand{\appendixtocname}{Prílohy}
  \renewcommand{\appendixname}{Príloha}
\fi
  \appendixpage

% vynechani stranky v oboustrannem rezimu
% Skip the page in the two-sided mode
\iftwoside
  \cleardoublepage
\fi
  
\ifslovak
%  \section*{Zoznam príloh}
%  \addcontentsline{toc}{section}{Zoznam príloh}
\else
  \ifczech
%    \section*{Seznam příloh}
%    \addcontentsline{toc}{section}{Seznam příloh}
  \else
%    \section*{List of Appendices}
%    \addcontentsline{toc}{section}{List of Appendices}
  \fi
\fi
  \startcontents[chapters]
  % seznam příloh / list of appendices
  % \printcontents[chapters]{l}{0}{\setcounter{tocdepth}{2}}
  
  % vynechani stranky v oboustrannem rezimu
  \iftwoside
    \cleardoublepage
  \fi
  % Tento soubor nahraďte vlastním souborem s přílohami (nadpisy níže jsou pouze pro příklad)
% This file should be replaced with your file with an appendices (headings below are examples only)

% Umístění obsahu paměťového média do příloh je vhodné konzultovat s vedoucím
% Placing of table of contents of the memory media here should be consulted with a supervisor
%\chapter{Obsah přiloženého paměťového média}

%\chapter{Manuál}

%\chapter{Konfigurační soubor} % Configuration file

%\chapter{RelaxNG Schéma konfiguračního souboru} % Scheme of RelaxNG configuration file

%\chapter{Plakát} % poster

\chapter{Jak pracovat s touto šablonou}
\label{jak}

V této kapitole je uveden popis jednotlivých částí šablony, po kterém následuje stručný návod, jak s touto šablonou pracovat. 

Jedná se o přechodnou verzi šablony. Nová verze bude zveřejněna do konce roku 2016 a bude navíc obsahovat nové pokyny ke správnému využití šablony, závazné pokyny k~vypracování bakalářských a diplomových prací (rekapitulace pokynů, které jsou dostupné na~webu) a nezávazná doporučení od vybraných vedoucích. Jediné soubory, které se v nové verzi změní, budou projekt-01-kapitoly-chapters.tex a projekt-30-prilohy-appendices.tex, jejichž obsah každý student vymaže a nahradí vlastním. Šablonu lze tedy bez problémů využít i~v~současné verzi.

\section*{Popis částí šablony}

Po rozbalení šablony naleznete následující soubory a adresáře:
\begin{DESCRIPTION}
  \item [bib-styles] Styly literatury (viz níže). 
  \item [obrazky-figures] Adresář pro Vaše obrázky. Nyní obsahuje placeholder.pdf (tzv. TODO obrázek, který lze použít jako pomůcku při tvorbě technické zprávy), který se s prací neodevzdává. Název adresáře je vhodné zkrátit, aby byl jen ve zvoleném jazyce.
  \item [template-fig] Obrázky šablony (znak VUT).
  \item [fitthesis.cls] Šablona (definice vzhledu).
  \item [Makefile] Makefile pro překlad, počítání normostran, sbalení apod. (viz níže).
  \item [projekt-01-kapitoly-chapters.tex] Soubor pro Váš text (obsah nahraďte).
  \item [projekt-20-literatura-bibliography.bib] Seznam literatury (viz níže).
  \item [projekt-30-prilohy-appendices.tex] Soubor pro přílohy (obsah nahraďte).
  \item [projekt.tex] Hlavní soubor práce -- definice formálních částí.
\end{DESCRIPTION}

Výchozí styl literatury (czechiso) je od Ing. Martínka, přičemž anglická verze (englishiso) je jeho překladem s drobnými modifikacemi. Oproti normě jsou v něm určité odlišnosti, ale na FIT je dlouhodobě akceptován. Alternativně můžete využít styl od Ing. Radima Loskota nebo od Ing. Radka Pyšného\footnote{BP Ing. Radka Pyšného \url{http://www.fit.vutbr.cz/study/DP/BP.php?id=7848}}. Alternativní styly obsahují určitá vylepšení, ale zatím nebyly řádně otestovány větším množstvím uživatelů. Lze je považovat za beta verze pro zájemce, kteří svoji práci chtějí mít dokonalou do detailů a neváhají si nastudovat detaily správného formátování citací, aby si mohli ověřit, že je vysázený výsledek v pořádku.

Makefile kromě překladu do PDF nabízí i další funkce:
\begin{itemize}
  \item přejmenování souborů (viz níže),
  \item počítání normostran,
  \item spuštění vlny pro doplnění nezlomitelných mezer,
  \item sbalení výsledku pro odeslání vedoucímu ke kontrole (zkontrolujte, zda sbalí všechny Vámi přidané soubory, a případně doplňte).
\end{itemize}

Nezapomeňte, že vlna neřeší všechny nezlomitelné mezery. Vždy je třeba manuální kontrola, zda na konci řádku nezůstalo něco nevhodného -- viz Internetová jazyková příručka\footnote{Internetová jazyková příručka \url{http://prirucka.ujc.cas.cz/?id=880}}.

\paragraph {Pozor na číslování stránek!} Pokud má obsah 2 strany a na 2. jsou jen \uv{Přílohy} a~\uv{Seznam příloh} (ale žádná příloha tam není), z nějakého důvodu se posune číslování stránek o 1 (obsah \uv{nesedí}). Stejný efekt má, když je na 2. či 3. stránce obsahu jen \uv{Literatura} a~je možné, že tohoto problému lze dosáhnout i jinak. Řešení je několik (od~úpravy obsahu, přes nastavení počítadla až po sofistikovanější metody). \textbf{Před odevzdáním proto vždy překontrolujte číslování stran!}


\section*{Doporučený postup práce se šablonou}

\begin{enumerate}
  \item \textbf{Zkontrolujte, zda máte aktuální verzi šablony.} Máte-li šablonu z předchozího roku, na stránkách fakulty již může být novější verze šablony s~aktualizovanými informacemi, opravenými chybami apod.
  \item \textbf{Zvolte si jazyk}, ve kterém budete psát svoji technickou zprávu (česky, slovensky nebo anglicky) a svoji volbu konzultujte s vedoucím práce (nebyla-li dohodnuta předem). Pokud Vámi zvoleným jazykem technické zprávy není čeština, nastavte příslušný parametr šablony v souboru projekt.tex (např.: \verb|documentclass[english]{fitthesis}| a přeložte prohlášení a poděkování do~angličtiny či slovenštiny.
  \item \textbf{Přejmenujte soubory.} Po rozbalení je v šabloně soubor projekt.tex. Pokud jej přeložíte, vznikne PDF s technickou zprávou pojmenované projekt.pdf. Když vedoucímu více studentů pošle projekt.pdf ke kontrole, musí je pracně přejmenovávat. Proto je vždy vhodné tento soubor přejmenovat tak, aby obsahoval Váš login a (případně zkrácené) téma práce. Vyhněte se však použití mezer, diakritiky a speciálních znaků. Vhodný název tedy může být např.: \uv{xlogin00-Cisteni-a-extrakce-textu.tex}. K přejmenování můžete využít i přiložený Makefile:
\begin{verbatim}
make rename NAME=xlogin00-Cisteni-a-extrakce-textu
\end{verbatim}
  \item Vyplňte požadované položky v souboru, který byl původně pojmenován projekt.tex, tedy typ, rok (odevzdání), název práce, svoje jméno, ústav (dle zadání), tituly a~jméno vedoucího, abstrakt, klíčová slova a další formální náležitosti.
  \item Nahraďte obsah souborů s kapitolami práce, literaturou a přílohami obsahem svojí technické zprávy. Jednotlivé přílohy či kapitoly práce může být výhodné uložit do~samostatných souborů -- rozhodnete-li se pro toto řešení, je doporučeno zachovat konvenci pro názvy souborů, přičemž za číslem bude následovat název kapitoly. 
  \item Nepotřebujete-li přílohy, zakomentujte příslušnou část v projekt.tex a příslušný soubor vyprázdněte či smažte. Nesnažte se prosím vymyslet nějakou neúčelnou přílohu jen proto, aby daný soubor bylo čím naplnit. Vhodnou přílohou může být obsah přiloženého paměťového média.
  \item Nascanované zadání uložte do souboru zadani.pdf a povolte jeho vložení do práce parametrem šablony v projekt.tex (\verb|documentclass[zadani]{fitthesis}|).
  \item Nechcete-li odkazy tisknout barevně (tedy červený obsah -- bez konzultace s vedoucím nedoporučuji), budete pro tisk vytvářet druhé PDF s tím, že nastavíte parametr šablony pro tisk: (\verb|documentclass[zadani,print]{fitthesis}|).  Barevné logo se nesmí tisknout černobíle!
  \item Vzor desek, do kterých bude práce vyvázána, si vygenerujte v informačním systému fakulty u zadání. Pro disertační práci lze zapnout parametrem v šabloně (více naleznete v souboru fitthesis.cls).
  \item Nezapomeňte, že zdrojové soubory i (obě verze) PDF musíte odevzdat na CD či jiném médiu přiloženém k technické zprávě.
\end{enumerate}

\subsection*{Pokyny pro oboustranný tisk}
\begin{itemize}
\item Zapíná se parametrem šablony: \verb|\documentclass[twoside]{fitthesis}|
\item Po vytištění oboustranného listu zkontrolujte, zda je při prosvícení sazební obrazec na obou stranách na stejné pozici. Méně kvalitní tiskárny s duplexní jednotkou mají často posun o 1--3 mm. Toto může být u některých tiskáren řešitelné tak, že vytisknete nejprve liché stránky, pak je dáte do stejného zásobníku a vytisknete sudé.
\item Za titulním listem, obsahem, literaturou, úvodním listem příloh, seznamem příloh a případnými dalšími seznamy je třeba nechat volnou stránku, aby následující část začínala na liché stránce (\textbackslash cleardoublepage).
\item  Konečný výsledek je nutné pečlivě překontrolovat.
\end{itemize}


\subsection*{Užitečné nástroje}
\label{nastroje}

Následující seznam není výčtem všech využitelných nástrojů. Máte-li vyzkoušený osvědčený nástroj, neváhejte jej využít. Pokud však nevíte, který nástroj si zvolit, můžete zvážit některý z následujících:

\begin{description}
	\item[\href{http://miktex.org/download}{MikTeX}] \LaTeX{} pro Windows -- distribuce s jednoduchou instalací a vynikající automatizací stahování balíčků.
	\item[\href{http://texstudio.sourceforge.net/}{TeXstudio}] Přenositelné opensource GUI pro \LaTeX{}.  Ctrl+klik umožňuje přepínat mezi zdrojovým textem a PDF. Má integrovanou kontrolu pravopisu, zvýraznění syntaxe apod. Pro jeho využití je nejprve potřeba nainstalovat MikTeX.
	\item[\href{http://jabref.sourceforge.net/download.php}{JabRef}] Pěkný a jednoduchý program v Javě pro správu souborů s bibliografií (literaturou). Není potřeba se nic učit -- poskytuje jednoduché okno a formulář pro editaci položek.
	\item[\href{https://inkscape.org/en/download/}{InkScape}] Přenositelný opensource editor vektorové grafiky (SVG i PDF). Vynikající nástroj pro tvorbu obrázků do odborného textu. Jeho ovládnutí je obtížnější, ale výsledky stojí za to.
	\item[\href{https://git-scm.com/}{GIT}] Vynikající pro týmovou spolupráci na projektech, ale může výrazně pomoci i jednomu autorovi. Umožňuje jednoduché verzování, zálohování a přenášení mezi více počítači.
	\item[\href{http://www.overleaf.com/}{Overleaf}] Online nástroj pro \LaTeX{}. Přímo zobrazuje náhled a umožňuje jednoduchou spolupráci (vedoucí může průběžně sledovat psaní práce), vyhledávání ve zdrojovém textu kliknutím do PDF, kontrolu pravopisu apod. Zdarma jej však lze využít pouze s určitými omezeními (někomu stačí na disertaci, jiný na ně může narazit i při psaní bakalářské práce) a pro dlouhé texty je pomalejší.
\end{description}

\subsection*{Užitečné balíčky pro \LaTeX}

Studenti při sazbě textu často řeší stejné problémy. Některé z nich lze vyřešit následujícími balíčky pro \LaTeX:

\begin{itemize}
  \item \verb|amsmath| -- rozšířené možnosti sazby rovnic,
  \item \verb|float, afterpage, placeins| -- úprava umístění obrázků,
  \item \verb|fancyvrb, alltt| -- úpravy vlastností prostředí Verbatim, 
  \item \verb|makecell| -- rozšíření možností tabulek,
  \item \verb|pdflscape, rotating| -- natočení stránky o 90 stupňů (pro obrázek či tabulku),
  \item \verb|hyphenat| -- úpravy dělení slov,
  \item \verb|picture, epic, eepic| -- přímé kreslení obrázků.
\end{itemize}

Některé balíčky jsou využity přímo v šabloně (v dolní části souboru fitthesis.cls). Nahlédnutí do jejich dokumentace může být rovněž užitečné.

Sloupec tabulky zarovnaný vlevo s pevnou šířkou je v šabloně definovaný \uv{L} (používá se jako \uv{p}).

 % viz. prilohy.tex / see prilohy.tex
\end{document}
