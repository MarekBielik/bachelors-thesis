\chapter{Conclusion}
This thesis demonstrated the capabilities of the evolutionary algorithms, namely the evolution strategies, in the domain of the analog amplifiers design. In the first phase, the task was to implement the concept of evolution strategies and integrate it with the ngSPICE simulator. An inherent part of the evolutionary algorithms is an appropriate fitness function so the next step was to develop methods for evaluating the quality of the amplifiers. In the last stage, there were various experiments carried out in order to demonstrate the perfromance of the proposed solution.

There were two types of amplifiers chosen for the optimization and it emerged that the evolution has the capability to find the desired solution and that it can even provide different variations of a circuit that has the same amplification properties. However, the results for the second amplifier were considerably limited.

There are various types of the fitness funcion developed and evaluated during the implementation and users can choose the most appropriate one according to their needs. The results of experiments were also used for the determination of the most optimal parameters for the evolution. During the experiments, it also emerged that various extensions to the basic version of evolutionary strategies had a great influence on the overall performance of the algorithm.

The resulting application provides a tool for designing amplifiers with arbitrary gain up to the maximum limits of the circuit without using any mathematical apparatus.

This thesis also contributed to the development of the ngSPICE simulator, which had memory leaks on various places in the source code. The simulator is run multiple times in a row during the optimization so there was a need to fix the memory leaks and the result is that they were successfully removed. The solution was proposed to the development team and the improvements will help in the next stages of development.

The future extension to this project could be an interface for entering various electronic circuits described in the SPICE syntax, so that users could utilize the current implementation of evolution strategies with different fitness functions and they would not be limited only to the embedded circuits.